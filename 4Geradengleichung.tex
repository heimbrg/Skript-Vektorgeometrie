%%%%%%%%%%%%%%% ÜBUNGEN SEITE 33


\begin{exercise}
$A=(2/1/4)$, $B=(5/3/5)$
\begin{enumerate}
\item Stellen Sie eine Gleichung der Geraden $g$ durch die Punkte $A$ und $B$ auf
\item Welcher der Punkte $P=(8/3/2)$, $Q=(-4/-3/2)$ und $R=(14/9/7)$ liegt auf $g$?
\end{enumerate}
\begin{answer}
\begin{enumerate}
\item $g: \vv{r} =\Vek{2}{1}{4} + t \Vek{3}{2}{1}$
\item $Q$ liegt auf $g$
\end{enumerate}
\end{answer}
\end{exercise}

\begin{exercise}
Nennen Sie die spezielle Lage bezüglich des Koordinatensystems der Geraden mit folgenden Gleichungen:

\begin{enumerate}
\item $\vv{r} = \Vek{0}{0}{5} + t \Vek{1}{4}{3}$
\item $\vv{r} = \Vek{1}{7}{2} + t \Vek{6}{-3}{0}$
\item $\vv{r} = \Vek{9}{1}{5} + t \Vek{1}{0}{0}$
\item $\vv{r} =  t \Vek{3}{-5}{1}$
\item $\vv{r} = \Vek{4}{0}{1} + t \Vek{-2}{0}{3}$
\item $\vv{r} = \Vek{2}{4}{6} + t \Vek{1}{2}{3}$
\item $\vv{r} = \Vek{0}{2}{0} + t \Vek{0}{-5}{0}$
\item $\vv{r} = \Vek{4}{2}{1} + t \Vek{2}{17}{3}$
\end{enumerate}

\begin{answer}
\begin{enumerate}
\item schneidet die $z$-Achse
\item verläuft parallel zur $xy$-Ebene
\item verläuft parallel zur $x$-Achse
\item geht durch den Ursprung
\item verläuft in der $xz$-Ebene
\item geht durch den Ursprung
\item fällt mit der $y$-Achse zusammen
\item keine spezielle Lage
\end{enumerate}
\end{answer}
\end{exercise}

\begin{exercise}
$g: \vv{r} = \Vek{1}{5}{-8} + t \Vek{1}{2}{5}$ \newline
Wenn Licht parallel zur $z$-Achse auf die Gerade $g$ fällt, entsteht als Schatten in der $xy$-Ebene eine Gerade $g'$. Man sagt, $g'$ sei die \textit{Normalprojektion} von $g$ auf die $xy$-Ebene. Bestimmen Sie eine Koordinatengleichung von $g'$!
\begin{answer}
Gleichungssystem: $x=1+t$ und $y=5+2t$, $t$ eliminieren $\rightarrow g': y=2x+3$
\end{answer}
\end{exercise}

\begin{exercise}
$g: \vv{r} = \Vek{3}{1}{10} + t \Vek{-1}{1}{-5}$ \newline
Bestimmen Sie die Spurpunkte der Geraden $g$. (Als die Spurpunkte einer Geraden bezeichnet man die Schnittpunkte dieser Geraden mit den Koordinatenachsen (in der Ebene) oder den Grundebenen des Koordinatensystems (im Raum).)
\begin{answer}
$S_{1} = (1/3/0)$, $S_{2} =(0/4/-5)$, $S_{3}=(4/0/15)$
\end{answer}
\end{exercise}

\begin{exercise}
Wie viele Spurpunkte kann eine Gerade haben? Diskutieren Sie alle möglichen Fälle.
\begin{answer}
In der Regel (Normalfall) drei Spurpunkte. Bei speziellen Lagen (Sonderfälle) ein, zwei oder unendlich viele Spurpunkte.
\end{answer}
\end{exercise}


\begin{exercise}
$S_{1} = (1/6/0)$, $S_{2} =(0/8/-2)$ \newline
Eine Gerade $g$ hat die beiden Spurpunkte $S_{1}$ und $S_{2}$. Bestimmen Sie den dritten Spurpunkt $S_{3}$!
\begin{answer}
Gleichung der Geraden durch $S_{1}$ und $S_{2}$ aufstellen $\rightarrow S_{3} = (4/0/6)$
\end{answer}
\end{exercise}

\begin{exercise}
$g: \vv{r} = \Vek{k-9}{0}{1} + t \Vek{2}{1}{1}$, $h: \vv{r} = \Vek{8}{7}{9} + t \Vek{4}{2}{k} $ \newline
Wie muss $k$ gewählt werden, damit die beiden Geraden $g$ und $h$
\begin{enumerate}
\item parallel sind?
\item sich schneiden?
\end{enumerate}
\begin{answer}
\begin{enumerate}
\item $k=2$ (echt parallele Geraden)
\item Parameter der Gleichungen von $g$ und $h$ verschieden bezeichnen: $t_{1}$ bzw. $t_{2}$, Gleichsetzen und Vektorgleichung nach $k$ auflösen $\rightarrow k=3 (t_{1} = 5 \text{ und } t_{2}=-1)$
\end{enumerate}
\end{answer}
\end{exercise}

\begin{exercise}
$P=(9/-1/8)$, $g : \vv{r} = \Vek{1}{1}{1} + t \Vek{3}{2}{2}$ \newline
Berechnen Sie den Abstand $d$ des Punktes $P$ von der Geraden $g$!
\begin{answer}
